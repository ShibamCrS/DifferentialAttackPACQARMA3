% !TEX root = ../truncated-reduced.tex
% !TEX TS-program = pdflatex
% !BIB TS-program = bibtex
% !LW recipe=pdflatex

%%%%%%%%%%%%%%%%%%%%%%%%%%%%%%%%%%%%%%%%%%%%%%%%%%%%%%%%%%%%%%%%%%%%%%%%%%%%%%%%
%
% Packages

\usepackage{etoolbox}
\usepackage{ifthen}
\usepackage{shellesc}

\usepackage{xstring}
\usepackage{array}
\usepackage{graphicx}
\usepackage[export]{adjustbox}
%\usepackage{rotating}
\usepackage{colortbl}

\usepackage{xspace}
\usepackage{empheq}
\usepackage{amsmath}
\usepackage{amssymb}
\usepackage{amsthm}

\usepackage{bigdelim}

\usepackage{doi}
%\usepackage{cite}
\usepackage[inline]{enumitem}

\usepackage{wasysym}
%\usepackage[mathlines]{lineno}
\usepackage[normalem]{ulem}

\usepackage{ragged2e}

%%%%%%%%%%%%%%%%%%%%%%%%%%%%%%%%%%%%%%%%%%%%%%%%%%%%%%%%%%%%%%%%%%%%%%%%%%%%%%%%%%%%%%%%%%%%%%%%%%%%%%%%%%%%%%%%%%%%%%%%%%%%%%%%
%
% To avoid having to define an extra environment just for isolated "local" claims

\newcommand{\thmpreskip}{\vskip\topsep}
\newcommand{\thmpostskip}{\vskip\topsep}

%%%%%%%%%%%%%%%%%%%%%%%%%%%%%%%%%%%%%%%%%%%%%%%%%%%%%%%%%%%%%%%%%%%%%%%%%%%%%%%%%%%%%%%%%%%%%%%%%%%%%%%%%%%%%%%%%%%%%%%%%%%%%%%%
%
%: Customisation of Tables

\usepackage{booktabs}
%\usepackage[list=true]{subcaption}
%\usepackage{longtable}
\usepackage{multirow}

%%%%%%%%%%%%%%%%%%%%%%%%%%%%%%%%%%%%%%%%%%%%%%%%%%%%%%%%%%%%%%%%%%%%%%%%%%%%%%%%%%%%%%%%%%%%%%%%%%%%%%%%%%%%%%%%%%%%%%%%%%%%%%%%
%
% Environments

\usepackage{environ}
\NewEnviron{ignore}{}
\newenvironment{DIFnomarkup}{}{}

%%%%%%%%%%%%%%%%%%%%%%%%%%%%%%%%%%%%%%%%%%%%%%%%%%%%%%%%%%%%%%%%%%%%%%%%%%%%%%%%%%%%%%%%%%%%%%%%%%%%%%%%%%%%%%%%%%%%%%%%%%%%%%%%
%
% Fonts and symbols

\usepackage{fontenc}

\usepackage{soul}

\usepackage[cal=esstix]{mathalpha}
\let\mycal\mathcal

\newcommand{\cmark}{\ding{51}}%
\newcommand{\xmark}{\ding{55}}%

\newcommand{\mapsfrom}{\mathrel{\reflectbox{\ensuremath{\mapsto}}}}

\newsavebox{\shortmapstobox}
\newdimen\shortmapstolen
\newcommand{\shortmapsto}{%
	\savebox{\shortmapstobox}{\hbox{\scalebox{0.9}[1]{\ensuremath{\mapsto}}}}%
	\shortmapstolen=\wd\shortmapstobox%
	\mathrel{\kern-0.333pt\usebox{\shortmapstobox}\hskip-\shortmapstolen\hskip0.0667pt\usebox{\shortmapstobox}\kern-0.333pt}}

\makeatletter
%\def\scriptsize{\@setfontsize{\scriptsize}{7}{8pt}}
\def\almosttiny{\@setfontsize{\almosttiny}{6}{7pt}}
%\def\tiny{\@setfontsize{\tiny}{5}{6pt}}
\makeatother

%\newcommand{\minimskip}{\mskip1mu}
\newcommand{\mm}{\mskip1mu}
\newcommand{\mmm}{\mskip-1mu}

\newcommand{\hm}{\hspace{1pt}}
\newcommand{\mhm}{\hspace{-1pt}}

\newcommand{\costq}{\mycal{E}_c}

\newcommand{\calA}{\mycal{A}}
\newcommand{\calB}{\mycal{B}}
\newcommand{\calC}{\mycal{C}}
\newcommand{\calD}{\mycal{D}}
\newcommand{\calE}{\mycal{E}}
\newcommand{\calf}{\mycal{f}}
\newcommand{\calg}{\mycal{g}}
\newcommand{\calH}{\mycal{H}}
\newcommand{\calI}{\mycal{I}}
\newcommand{\calJ}{\mycal{J}}
\newcommand{\calK}{\mycal{K}}
\newcommand{\calL}{\mycal{L}}
\newcommand{\calR}{\mycal{R}}
\newcommand{\calS}{\mycal{S}}
\newcommand{\calG}{\mathcal{G}}

\newcommand{\scrI}{\Im}
\newcommand{\clusterp}{{\check p}}
\newcommand{\tweaks}{\mycal{T}}
\newcommand{\F}{\mathbb{F}}
\newcommand{\ie}{i.e.\ }
\newcommand{\AES}{\texttt{AES}\xspace}
\newcommand{\DES}{\texttt{DES}\xspace}
\newcommand{\ASCON}{\texttt{ASCON}\xspace}
\newcommand{\SHA}{\texttt{SHA}\xspace}
\newcommand{\GCM}{\texttt{GCM}\xspace}
\newcommand{\QARMA}{\texttt{QARMA}\xspace}
\newcommand{\qarma}{\texttt{QARMA}\xspace}
\newcommand{\QARMAvii}{\texttt{QARMAv2}\xspace}
\newcommand{\SKINNY}{\texttt{SKINNY}\xspace}
\newcommand{\qarmavii}{\texttt{QARMAv2}\xspace}
\newcommand{\qameleon}{\texttt{Qameleon}\xspace}
\newcommand{\PRINCE}{\texttt{PRINCE}\xspace}
\newcommand{\GIFT}{\texttt{GIFT}\xspace}
\newcommand{\prince}{\texttt{PRINCE}\xspace}
\newcommand{\PRINCEvtwo}{\texttt{PRINCEv2}\xspace}
\newcommand{\PRESENT}{\texttt{PRESENT}\xspace}
\newcommand{\MANTIS}{\texttt{MANTIS}\xspace}
\newcommand{\DeoxysBC}{\texttt{Deoxys-BC}\xspace}
\newcommand{\MIDORI}{\texttt{MIDORI}\xspace}
\newcommand{\ENIGMA}{\texttt{ENIGMA}\xspace}
\newcommand{\djbPoly}{\texttt{Poly1305}\xspace}
\newcommand{\ChaChaTw}{\texttt{ChaCha20}\xspace}
\newcommand{\SPEEDY}{\texttt{SPEEDY}\xspace}
\newcommand{\gemfive}{\texttt{gem5}\xspace}
\newcommand{\clefia}{\texttt{CLEFIA}\xspace}
\newcommand{\katan}{\texttt{KATAN}\xspace}
\newcommand{\ktantan}{\texttt{KTANTAN}\xspace}
\newcommand{\klein}{\texttt{KLEIN}\xspace}
\newcommand{\led}{\texttt{LED}\xspace}
\newcommand{\mercy}{\texttt{MERCY}\xspace}
\newcommand{\threefish}{\texttt{THREEFISH}\xspace}
\newcommand{\skein}{\texttt{SKEIN}\xspace}
\newcommand{\TWEAKEY}{\texttt{TWEAKEY}\xspace}
\newcommand{\Pholkos}{\texttt{Pholkos}\xspace}
\newcommand{\SIPHASH}{\texttt{SIPHASH}\xspace}

\newcommand{\simon}{\texttt{SIMON}\xspace}
\newcommand{\speck}{\texttt{SPECK}\xspace}
\newcommand{\skinny}{\texttt{SKINNY}\xspace}
\newcommand{\mantis}{\texttt{MANTIS}\xspace}
\newcommand{\midori}{\texttt{MIDORI}\xspace}
\newcommand{\present}{\texttt{PRESENT}\xspace}
\newcommand{\blowfish}{\texttt{Blowfish}\xspace}

\newcommand{\IS}{\mathrm{IS}}
\newcommand{\SC}{\texttt{StateShuffle}\xspace} % TODO confusing naming (short name: SC = SubCells; long name: should start with verb (ShuffleState)).
\newcommand{\MCol}{\texttt{MixColumns}\xspace}
\newcommand{\MR}{\texttt{MixRows}\xspace}
\newcommand{\XR}{\texttt{eXchangeRows}\xspace}
\newcommand{\orto}{\mycal o}
\newcommand{\ORTO}{\mycal O}
\newcommand{\idplusorto}{\mycal p}

\newcommand{\mono}{\textsf{mono}\xspace}
\newcommand{\orthros}{\textsf{Orthros}\xspace}
\newcommand{\splitcounter}{\textsf{split}\xspace}
\newcommand{\oCC}{\textsf{oCC}\xspace}
\newcommand{\LoC}{\textsf{LoC}\xspace}
\newcommand{\BoC}{\textsf{BoC}\xspace}
\newcommand{\oCCnL}{\textsf{oCC-nL}\xspace}
\newcommand{\MirE}{\textsf{MirE}\xspace}
\newcommand{\Lzero}{\textsf{L0}\xspace}
\newcommand{\LI}{\textsf{L1}\xspace}
\newcommand{\LII}{\textsf{L2}\xspace}
\newcommand{\LIminus}{\textsf{L1--}\xspace}
\newcommand{\LIIminus}{\textsf{L2--}\xspace}
\newcommand{\LIIp}{\textsf{L2+}\xspace}
\newcommand{\LIII}{\textsf{L3}\xspace}
\newcommand{\SGX}{\textsf{SGX}\xspace}
\newcommand{\CCA}{\textsf{CCA}\xspace}
\newcommand{\TDX}{\textsf{TDX}\xspace}
\newcommand{\SEV}{\textsf{SEV}\xspace}
\newcommand{\SME}{\textsf{SME}\xspace}
\newcommand{\cfgvec}[3]{\textsf{#1/\texttt{#2}/#3}\xspace}
\newcommand{\XOR}{\texttt{XOR}\xspace}
\newcommand{\GE}{\textsf{GE}\xspace}

\newcommand{\deltain}{\delta_{\mathit{in}}}
%\newcommand{\deltaT}{\delta_{\mathrm{T}}}
\newcommand{\deltaout}{\delta_{\kern-.055em\mathit{out}}}
\newcommand{\Deltain}{\Delta_{\mathit{in}}}
\newcommand{\Deltaout}{\Delta_{\kern-.027em\mathit{out}}}
\newcommand{\Nablain}{\nabla_{\mathit{in}}}
\newcommand{\Nablaout}{\nabla_{\kern-.027em\mathit{out}}}
\newcommand{\DeltaT}{\Delta_{\mathrm{T}}}
\newcommand{\din}{d_{\mathit{in}}}
\newcommand{\dout}{d_{\kern-.027em\mathit{out}}}
\newcommand{\ein}{e_{\mathit{in}}}
\newcommand{\eout}{e_{\kern-.027em\mathit{out}}}
\newcommand{\dinstar}{\din^{\kern.04em*}}
\newcommand{\dT}{d_{\kern-.027em\mathit{T}}}
\newcommand{\hatdout}{{\widehat d}_{\kern-.027em\mathit{out}}}
\newcommand{\Din}{D_{\mathit{in}}}
\newcommand{\Dout}{D_{\kern-.11em\mathit{out}}}
\newcommand{\DT}{D_{\kern-.027em\mathit{T}}}
\newcommand{\hatDout}{{\widehat D}_{\kern-.11em\mathit{out}}}

\newcommand{\smin}{s_{\mathit{min}}}
\newcommand{\zetamin}{\zeta_{\mathit{min}}}

\newcommand{\rin}{r_{\mathit{in}}}
\newcommand{\rout}{r_{\kern-.11em\mathit{out}}}
\newcommand{\rmid}{r_{\mathit{mid}}}
\newcommand{\Kbits}{\calK}
\newcommand{\Kin}{\Kbits_{\mathit{in}}}
\newcommand{\Kout}{\Kbits_{\kern-.082em\mathit{out}}}
\newcommand{\keyguess}{k}
\newcommand{\kin}{\keyguess_{\mathit{in}}}
\newcommand{\kout}{\keyguess_{\kern-.055em\mathit{out}}}
\newcommand{\cin}{c_{\mathit{in}}}
\newcommand{\cout}{c_{\kern-.055em\mathit{out}}}

\newcommand{\TGF}{T_{\textit{GF}}}
\newcommand{\SGF}{S_{\textit{GF}}}
\newcommand{\TMR}{T_{\textit{MR}}}
\newcommand{\TMW}{T_{\textit{MW}}}
\newcommand{\Tenc}{T_{\calE}}
\newcommand{\Tcollect}{T_{\textit{C}}}
% \newcommand{\Tcollect}{T_{\calE}}

\DeclareFontFamily{U}{rcjhbltx}{}
\DeclareFontShape{U}{rcjhbltx}{m}{n}{<->rcjhbltx}{}
\DeclareSymbolFont{hebrewletters}{U}{rcjhbltx}{m}{n}

% remove the definitions from amssymb
\let\aleph\relax\let\beth\relax
\let\gimel\relax\let\daleth\relax

\DeclareMathSymbol{\aleph}{\mathord}{hebrewletters}{39}
\DeclareMathSymbol{\beth}{\mathord}{hebrewletters}{98}\let\bet\beth
\DeclareMathSymbol{\gimel}{\mathord}{hebrewletters}{103}
\DeclareMathSymbol{\daleth}{\mathord}{hebrewletters}{100}\let\dalet\daleth

\DeclareMathSymbol{\lamed}{\mathord}{hebrewletters}{108}
\DeclareMathSymbol{\mem}{\mathord}{hebrewletters}{109}\let\mim\mem
\DeclareMathSymbol{\ayin}{\mathord}{hebrewletters}{96}
\DeclareMathSymbol{\tsadi}{\mathord}{hebrewletters}{118}
\DeclareMathSymbol{\qof}{\mathord}{hebrewletters}{113}
\DeclareMathSymbol{\shin}{\mathord}{hebrewletters}{152}

\def\hpzz{\hphantom{.00}}
\def\hzz{\hphantom{00}}
\def\hz{\hphantom{0}}
\def\lbar{\mbox{\l}}

%\def\linenumberfont{\normalfont\tiny\sffamily\color{gray}}
%\setcounter{tocdepth}{3}
%\def\colornopagebreak#1{\color{#1}\nopagebreak}

%%%%%%%%%%%%%%%%%%%%%%%%%%%%%%%%%%%%%%%%%%%%%%%%%%%%%%%%%%%%%%%%%%%%%%%%%%%%%%%%
%
% Colors

\usepackage{xcolor}

\selectcolormodel{rgb}

\definecolor{lightred}{rgb}{0.8,0.2,0.2}
\definecolor{darkred}{rgb}{0.75,0,0}

\definecolor{darkblue}{rgb}{0,0.4,0.8}
\definecolor{armBlue}{rgb}{0, 0.55859375, 0.73828125}
\definecolor{armMidBlue}{rgb}{0, 0.36328125, 0.51171875}
\definecolor{armDarkBlue}{rgb}{0, 0.216796875, 0.3984375}
\definecolor{armDarkerBlue}{rgb}{0, 0.16796875, 0.28515625}

\definecolor{verydarkgray}{rgb}{.13,.13,.13}
\definecolor{darkgray}{rgb}{.4,.4,.4}
\definecolor{midgray}{rgb}{.5,.5,.5}
\definecolor{gray}{rgb}{.5,.5,.5}
\definecolor{lightgray}{rgb}{.75,.75,.75}
\definecolor{verylightgray}{rgb}{.93,.93,.93}
\definecolor{lightestgray}{rgb}{.97,.97,.97}

\definecolor{lightgreen}{rgb}{0.16,0.8,0.48}
\definecolor{midgreen}{rgb}{0.13,0.625,0.33}
\definecolor{darkgreen}{rgb}{0.1,0.48,0.16}
\definecolor{darkergreen}{rgb}{0.075,0.32,0}
\definecolor{armGreen}{rgb}{0.171875,0.765625,0.4453125}

\definecolor{joli}{RGB}{225,95,0}
\definecolor{cupo}{RGB}{112,47,0}
\definecolor{dandelion}{HTML}{FDBC42}
\definecolor{mustard}{HTML}{FFDB58}
\definecolor{darkorange}{rgb}{1.00,0.40,0.00}

\definecolor{tug}{HTML}{F70146}
\colorlet   {colA}{tug}%          % Akzent 1
\definecolor{colB}{HTML}{5191C1}% % Akzent 2
\definecolor{colC}{HTML}{A5A5A5}% % Akzent 3
\definecolor{colD}{HTML}{285F82}% % Akzent 4
\definecolor{colE}{HTML}{78B473}% % Akzent 5
\definecolor{colF}{HTML}{E59352}% % Akzent 6

%%% TUG named palette
\colorlet{tugred}{colA}
\colorlet{tuggreen}{colE}
\colorlet{tugblue}{colD}
\colorlet{tugyellow}{colF}

\colorlet{diff}{tugblue}
\colorlet{marc}{tugyellow}
\colorlet{cost}{tugred}
\colorlet{state}{tugblue}

\selectcolormodel{cmyk}

\def\legendcolor#1{#1}

\def\redtext#1{\textcolor{red}{#1}}
\def\bluetext#1{\textcolor{blue}{#1}}

\makeatletter
\newcommand{\xmapsfrom}[2][]{%
	\ext@arrow3095\leftarrowfill@{#1}{#2}\mapsfromchar
}
\makeatother

\def\graymidrule{\arrayrulecolor{lightgray}\midrule\arrayrulecolor{black}}

\newdimen\myaboverulesep
\newdimen\mybelowrulesep
\myaboverulesep=0.45ex
\mybelowrulesep=0.70ex
\makeatletter
\def\mymidrule{\noalign{\ifnum0=`}\fi
	\@aboverulesep=\myaboverulesep
	\global\@belowrulesep=\mybelowrulesep
	\global\@thisruleclass=\@ne
	\@ifnextchar[{\@BTrule}{\@BTrule[\lightrulewidth]}}
\makeatother

\def\mygraymidrule{\arrayrulecolor{lightgray}\mymidrule\arrayrulecolor{black}}

%%%%%%%%%%%%%%%%%%%%%%%%%%%%%%%%%%%%%%%%%%%%%%%%%%%%%%%%%%%%%%%%%%%%%%%%%%%%%%%%
%
% References, back references, links, acronyms

%\usepackage[nospace]{varioref}

\makeatletter
\@ifundefined{DIFadd}{
	\AtEndPreamble{
		\usepackage[hyperpageref]{backref}
		\renewcommand*{\backref}[1]{}
		\renewcommand*{\backrefalt}[4]{%
			\ifcase #1 % No citations.%
			\or
				%\hfill\break
				Cited on page~#2.%
			\else
				%\hfill\break
				Cited on pages~#2.%
			\fi}
	}
}{}
\makeatother

\AtEndPreamble{%
	\usepackage[capitalise]{cleveref}

	\newtheorem{defn}{Definition}[section]
	\newtheorem{defns}[defn]{Definitions}
	\newtheorem{rem}[defn]{Remark}

	\crefname{subsection}{Section}{Sections}
	\Crefname{subsection}{Section}{Sections}
	\crefname{subsubsection}{Section}{Sections}
	\Crefname{subsubsection}{Section}{Sections}
	\crefname{rem}{Remark}{Remarks}
	\Crefname{rem}{Remark}{Remarks}
	\crefname{equation}{Equation}{Equations}
	\Crefname{equation}{Equation}{Equations}
	\crefname{figure}{Figure}{Figures}
	\Crefname{table}{Table}{Tables}
	\crefname{table}{Table}{Tables}

	%%% hack to sidestep a name collision
	\let\saveComment\Comment
	\let\saveQarmaState\State
	\let\State\relax
	\usepackage[ruled]{algorithm}
	\usepackage[noend]{algpseudocode}
	\let\AlgoState\State
	\let\AlgoComment\Comment
	\let\AlgoStatex\Statex

	\let\State\MatrixState
	\let\Comment\saveComment
}

\newcommand{\this}[1]{#1\xspace}
\newcommand{\myvskip}{\vskip 0.5\topsep}%
\newcommand{\bit}{-bit\xspace}
\newcommand{\bits}{\,bits\xspace}

%%%%%%%%%%%%%%%%%%%%%%%%%%%%%%%%%%%%%%%%%%%%%%%%%%%%%%%%%%%%%%%%%%%%%%%%%%%%%%%%%%%%%%%%%%%%%%%%%%%%%%%%%%%%%%%%%%%%%%%%%%%%%%%%
%
% Lengths
%
\newlength\replength
\newcommand\ruleht{2.5pt}% ELEVATION OF RULE
\newcommand\repfrac{1}% SOLID FRACTION OF DASH LINE [0->1] (USE 1 FOR SOLID)
\replength=.6em\relax% PERIOD OF DASHED RULE
\newcommand\rulewidth{0.5pt}% THICKNESS OF RULE
\newcommand\drulefill{\leavevmode\dashfill\hfil%
	\kern\dimexpr\repfrac\replength-\replength\relax}
\newcommand\dashfill[1][\repfrac]{\cleaders\hbox to \replength{%
		\smash{\rule[\ruleht]{\repfrac\replength}{\rulewidth}}}\hfill}

%%%%%%%%%%%%%%%%%%%%%%%%%%%%%%%%%%%%%%%%%%%%%%%%%%%%%%%%%%%%%%%%%%%%%%%%%%%%%%%%%%%%%%%%%%%%%%%%%%%%%%%%%%%%%%%%%%%%%%%%%%%%%%%%
%
% Dirty fix for misalignment of tags

\makeatletter
\def\place@tag{%
	\kern-\tagshift@
	\if1\shift@tag\row@\relax
		\llap{\vtop{%
				\raise@tag%
				\normalbaselines%
				\setbox\@ne\null%
				\dp\@ne\lineht@
				\box\@ne%
				\hbox{\boxz@\hskip-3.5pt}%
			}}%
	\else
		\llap{\boxz@}%
	\fi
}
\makeatother

%%%%%%%%%%%%%%%%%%%%%%%%%%%%%%%%%%%%%%%%%%%%%%%%%%%%%%%%%%%%%%%%%%%%%%%%%%%%%%%%%%%%%%%%%%%%%%%%%%%%%%%%%%%%%%%%%%%%%%%%%%%%%%%%

\def\naively{na\"\i vely\xspace}
\def\Naively{Na\"\i vely\xspace}

%%%%%%%%%%%%%%%%%%%%%%%%%%%%%%%%%%%%%%%%%%%%%%%%%%%%%%%%%%%%%%%%%%%%%%%%%%%%%%%%
%
% my special \wbar

%\usepackage{amsmath}
\makeatletter
\let\save@mathaccent\mathaccent
\newcommand*\if@single[3]{%
\setbox0\hbox{${\mathaccent"0362{#1}}^H$}%
\setbox2\hbox{${\mathaccent"0362{\kern0pt#1}}^H$}%
\ifdim\ht0=\ht2 #3\else #2\fi
}
%The bar will be moved to the right by a half of \macc@kerna, which is computed by amsmath:
\newcommand*\rel@kern[1]{\kern#1\dimexpr\macc@kerna}
%If there's a superscript following the bar, then no negative kern may follow the bar;
%an additional {} makes sure that the superscript is high enough in this case:
\newcommand*\widebar[1]{\@ifnextchar^{{\wide@bar{#1}{0}}}{\wide@bar{#1}{1}}}
%Use a separate algorithm for single symbols:
\newcommand*\wide@bar[2]{\if@single{#1}{\wide@bar@{#1}{#2}{1}}{\wide@bar@{#1}{#2}{2}}}
\newcommand*\wide@bar@[3]{%
	\begingroup
	\def\mathaccent##1##2{%
		%Enable nesting of accents:
		\let\mathaccent\save@mathaccent
		%If there's more than a single symbol, use the first character instead (see below):
		\if#32 \let\macc@nucleus\first@char \fi
		%Determine the italic correction:
		\setbox\z@\hbox{$\macc@style{\macc@nucleus}_{}$}%
		\setbox\tw@\hbox{$\macc@style{\macc@nucleus}{}_{}$}%
		\dimen@\wd\tw@
		\advance\dimen@-\wd\z@
		%Now \dimen@ is the italic correction of the symbol.
		\divide\dimen@ 3
		\@tempdima\wd\tw@
		\advance\@tempdima-\scriptspace
		%Now \@tempdima is the width of the symbol.
		\divide\@tempdima 10
		\advance\dimen@-\@tempdima
		%Now \dimen@ = (italic correction / 3) - (Breite / 10)
		\ifdim\dimen@>\z@ \dimen@0pt\fi
		%The bar will be shortened in the case \dimen@<0 !
		\rel@kern{0.6}\kern-\dimen@
		\if#31
			\overline{\rel@kern{-0.6}\kern\dimen@\macc@nucleus\rel@kern{0.4}\kern\dimen@}%
			\advance\dimen@0.4\dimexpr\macc@kerna
			%Place the combined final kern (-\dimen@) if it is >0 or if a superscript follows:
			\let\final@kern#2%
			\ifdim\dimen@<\z@ \let\final@kern1\fi
			\if\final@kern1 \kern-\dimen@\fi
		\else
			\overline{\rel@kern{-0.6}\kern\dimen@#1}%
		\fi
	}%
	\macc@depth\@ne
	\let\math@bgroup\@empty \let\math@egroup\macc@set@skewchar
	\mathsurround\z@ \frozen@everymath{\mathgroup\macc@group\relax}%
	\macc@set@skewchar\relax
	\let\mathaccentV\macc@nested@a
	%The following initialises \macc@kerna and calls \mathaccent:
	\if#31
		\macc@nested@a\relax111{#1}%
	\else
		%If the argument consists of more than one symbol, and if the first token is
		%a letter, use that letter for the computations:
		\def\gobble@till@marker##1\endmarker{}%
		\futurelet\first@char\gobble@till@marker#1\endmarker
		\ifcat\noexpand\first@char A\else
			\def\first@char{}%
		\fi
		\macc@nested@a\relax111{\first@char}%
	\fi
	\endgroup
}
\makeatother

\let\bar\widebar
\let\mybar\widebar

%%%%%%%%%%%%%%%%%%%%%%%%%%%%%%%%%%%%%%%%%%%%%%%%%%%%%%%%%%%%%%%%%%%%%%%%%%%%%%%%%%%%%%%%%%%%%%%%%%%%%%%%%%%%%%%%%%%%%%%%%%%%%%%%
%%%%%%%%%%%%%%%%%%%%%%%%%%%%%%%%%%%%%%%%%%%%%%%%%%%%%%%%%%%%%%%%%%%%%%%%%%%%%%%%%%%%%%%%%%%%%%%%%%%%%%%%%%%%%%%%%%%%%%%%%%%%%%%%
%%%%%%%%%%%%%%%%%%%%%%%%%%%%%%%%%%%%%%%%%%%%%%%%%%%%%%%%%%%%%%%%%%%%%%%%%%%%%%%%%%%%%%%%%%%%%%%%%%%%%%%%%%%%%%%%%%%%%%%%%%%%%%%%
